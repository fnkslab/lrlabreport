%%%%%%
%
%  @@@@@@@ related-work template @@@@@@
%  Use this template for compiling related work notes.
%  Unlike lrlab-report, you do not need to create a tex file for each reporting event.
%  Keep updating one tex file for your research theme.%
%
%%%%%%
%% Set the compiler to LaTeX on Overleaf
%%%%%%
%% Comment out one of the two \documentclass lines
%%%%%%
\documentclass[slide]{lrlabreport}
\date{\bf Related-Work Notes for:} % No need to edit this definition
\renewcommand{\refname}{参考文献一覧/Bibliography}

\usepackage{lrlabreport} % lrlabreport.sty
\newbibfield{title}
\newbibfield{author}
\newbibfield{journal}
\newbibfield{booktitle}
\newbibfield{url}
\newbibfield{doi}
\newbibfield{abstract}
\bibinput{sample} % required by usebib.sty called in lrlabreport.sty


%%%%%%%%%%%%%%%%%%%%%%%%%%%%%%%%%%%%%%%%
% Paper information commands
% (Customize as you like)
%%%%%%%%%%%%%%%%%%%%%%%%%%%%%%%%%%%%%%%%
\newcommand{\jour}[1]{%
{\footnotesize
\begin{description}
\item[Title:] {\color{blue}\bf \usebibentry{#1}{title}}
\item[Author(s):] \usebibentry{#1}{author}
\item[Journal:] \usebibentry{#1}{journal}
\item[DOI:] \url{http://doi.org/\usebibentry{#1}{doi}}
\end{description}
}
}

\newcommand{\conf}[1]{%
{\footnotesize
\begin{description}
\item[Title:] {\color{blue}\bf\usebibentry{#1}{title}}
\item[Author(s):] \usebibentry{#1}{author}
\item[Conference:] \usebibentry{#1}{booktitle}
\item[URL:] \usebibentryurl{#1}
\end{description}
}
}

\newcommand{\abst}[1]{%
{\footnotesize
Abstract: \usebibentry{#1}{abstract}
}
}

%%%%%%%%%%%%%%%%%%%%%%%%%%%%%%%%%%%%%%%%%%%%%%%%%%%%%%%%%%%%%%%%%%%%%%%

% define here your commands as you want
% E.g., \newcommand{\eg}{\textit{e.g.,}}

%%%%%%%%%%%%%%%%%%%%%%%%%%%%%%%%%%%%%%%
%%% title page content
%%%%%%%%%%%%%%%%%%%%%%%%%%%%%%%%%%%%%%%

\title{Your Research Theme}

\author{Student's name in both Latin alphabet and 漢字 (if you have)}

\begin{document}

\maketitle
\begin{abstract}
\noindent
    Summarize your research theme, motivation, and goal.
\end{abstract}
\thispagestyle{empty}

%%%%%%%%%%%%%%%%%%%%%%%%%%%%%%%%%%%%%%%
%%% body content
%%%%%%%%%%%%%%%%%%%%%%%%%%%%%%%%%%%%%%%

\section{Category 1}

\I Use \verb|\section| to categorize references
\I Describe the category briefly below \verb|\section| like this

\subsection{\cite{devlin-etal-2019-bert}}
\conf{devlin-etal-2019-bert}
\I Use \verb|subsection| for describing each paper
\I Leave any note for you here
\I It is good to note the source of the reference information (e.g., ``from Professor'', ``found accidentally'', ``cited by \cite{nakaneNLP2025}'', etc.)

\subsection{\cite{PLV}}
\jour{PLV}
\I Use \verb|\conf| to show conference paper information
\I A conference paper should have a \verb|url| field
\I Use \verb|\jour| to show journal paper information
\I A conference paper should has a \verb|doi| field

\section{Category 2}

\I The second group example
\I If you do not need categories, just remove \verb|\section|
\I As you use \verb|\subsection| for papers, categories cannot be hierarchical (flat category structure).

\subsection{\cite{nakaneNLP2025}}
\conf{nakaneNLP2025}
\abst{nakaneNLP2025} 

\I If you want to show abstract as above, use \verb|\abst|

\footnotesize
\bibliographystyle{apalike}
\bibliography{sample} % should be the same or include the bib file for \bibinput

\end{document}
